\documentclass[journal,12pt,twocolumn]{IEEEtran}

\usepackage{setspace}
\usepackage{amssymb}
\usepackage[cmex10]{amsmath}
\usepackage{longtable}
\usepackage{mathrsfs}
\usepackage{float}
\usepackage{multirow}
\usepackage{enumitem}
\usepackage{mathtools}
\usepackage{tfrupee}
\usepackage[breaklinks=true]{hyperref}
\usepackage{listings}
    \usepackage{color}                                            %%
    \usepackage{array}                                            %%
    \usepackage{longtable}                                        %%
    \usepackage{calc}                                             %%
    \usepackage{multirow}                                         %%
    \usepackage{hhline}                                           %%
    \usepackage{ifthen}                                           %%
    \usepackage{lscape}     
\DeclareMathOperator*{\equals}{=}

\renewcommand\thesectiondis{\arabic{section}}
\renewcommand\thesubsectiondis{\thesectiondis.\arabic{subsection}}
\renewcommand\thesubsubsectiondis{\thesubsectiondis.\arabic{subsubsection}}

\def\inputGnumericTable{}                                 %%
\lstset{
%language=C,
frame=single, 
breaklines=true,
columns=fullflexible
}
\bibliographystyle{IEEEtran}
\providecommand{\pr}[1]{\ensuremath{\Pr\left(#1\right)}}
\providecommand{\brak}[1]{\ensuremath{\left(#1\right)}}
\providecommand{\cbrak}[1]{\ensuremath{\left\{#1\right\}}}
\newcommand{\mydet}[1]{\ensuremath{\begin{vmatrix}#1\end{vmatrix}}}
\newcommand{\myvec}[1]{\ensuremath{\begin{pmatrix}#1\end{pmatrix}}}
\newcommand*{\permcomb}[4][0mu]{{{}^{#3}\mkern#1#2_{#4}}}
\newcommand*{\perm}[1][-3mu]{\permcomb[#1]{P}}
\newcommand*{\comb}[1][-1mu]{\permcomb[#1]{C}}

\newcommand{\question}{\noindent \textbf{Question: }}
\newcommand{\solution}{\noindent \textbf{Solution: }}

\begin{document}
\title{ASSIGNMENT 4}
\author{AKHILA, CS21BTECH11031}

\maketitle
\question
Find the probability that when a hand of 7 cards is drawn from a well
shuffled deck of 52 cards, it contains
    \begin{enumerate}[label=(\roman{enumi})]
		\item all Kings
		\item 3 Kings
		\item atleast 3 Kings
    \end{enumerate}
\solution
Given that a hand of 7 cards is drawn from a well shuffled deck of 52 cards.\\
Total number of possible hands $=\myvec{52 \\ 7}$\\

Let's a random variable $X$ such that \\
$X\in \cbrak{0,1,2,3,4}$ denote the outcome of the given problem.
\begin{table}[H]
\input{table.tex}
	\caption{Random variable and Event distribution}
	\label{tab1}
\end{table}
\begin{enumerate}[label=(\roman{enumi})]
      \item Number of hands with 4 Kings $=\myvec{4 \\ 4}\times \myvec{48 \\ 3}$\\
      (Since the other 3 cards must be choosen from remaining 48 cards.)\\
      The probability that hand of 7 cards contains all the kings is
\begin{align}
		 &\pr{X=4}=\frac{\myvec{4 \\ 4}\times \myvec{48 \\ 3}}{\myvec{52 \\ 7}}\\
		 &=\frac{1}{7735}
\end{align}
      \item Number of hands with 3 Kings $=\myvec{4 \\ 3}\times\myvec{48 \\ 4}$\\
      (Since the other 4 cards must be choosen from remaining 48 cards.)\\
      The probability that hand of 7 cards contains 3 kings is
\begin{align}
		 &\pr{X=3}=\frac{\myvec{4 \\ 3}\times \myvec{48 \\ 4}}{\myvec{52 \\ 7}}\\
		 &=\frac{9}{1547}
\end{align}
      \item The probability that hand of 7 cards contains atleast 3 kings is
\begin{align}
     &\pr{X\geq 3}=\pr{X=3}+\pr{X=4}\\
	 &=\frac{1}{7735}+\frac{9}{1547}\\
	 &=\frac{46}{7735}
\end{align}
\end{enumerate}
\end{document}